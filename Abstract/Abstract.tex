\documentclass[english,a4paper,oneside, onecolumn,article,9pt]{memoir}
\usepackage[english]{babel}		
\usepackage[utf8x]{inputenc}
\usepackage{microtype}
\usepackage{graphicx}
\usepackage{amsmath}
\usepackage{amssymb}
\usepackage{amsfonts}
\usepackage{mathtools}
\usepackage{siunitx}
\usepackage{xspace}
\usepackage{mathrsfs}
\usepackage{slashed}
\usepackage[inline]{enumitem}
\usepackage[colorlinks=true,linkcolor=black]{hyperref}
\usepackage{cleveref}

%\setlength\parindent{0pt}


\title{Abstract}
\author{Malthe Kjær Bisbo}
\date{\vspace{-5ex}}
\begin{document}
\thispagestyle{empty}
\maketitle

To reduce the runtime of structure optimization on computationally expensive DFT based potential energy surfaces (PES's), this project is about using machine learning to generate a computationally cheep model PES on which to perform global structure optimization. During structure optimisation the model PES is continuously updated, and thus refined, based on frequent samples of the DFT based PES.

\end{document}