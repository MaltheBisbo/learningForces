\documentclass[english,a4paper,oneside, twocolumn,article,9pt]{memoir}
\usepackage[english]{babel}		
\usepackage[utf8x]{inputenc}
\usepackage{microtype}
\usepackage{graphicx}
\usepackage{amsmath}
\usepackage{amssymb}
\usepackage{amsfonts}
\usepackage{mathtools}
\usepackage{siunitx}
\usepackage{xspace}
\usepackage{mathrsfs}
\usepackage{slashed}
\usepackage[inline]{enumitem}
\usepackage[colorlinks=true,linkcolor=black]{hyperref}
\usepackage{cleveref}

\setlength\parindent{0pt}

\newcommand{\mb}[1]{\mathbf{#1}}

\newcommand{\deriv}[1]{% \deriv[<order>]{<func>}{<var>}
	\ensuremath{\frac{\partial}{\partial {#1}}}}

\newcommand{\Deriv}[3][]{% \deriv[<order>]{<func>}{<var>}
	\ensuremath{\frac{\partial^{#1} {#2}}{\partial {#3}^{#1}}}}

\newcommand{\ket}[1]{\vert #1 \rangle}
\newcommand{\bra}[1]{\langle#1\vert}
\newcommand{\LL}{\ensuremath{\mathcal{L}}}


\newcommand{\norm}[1]{\left\lVert#1\right\rVert}

\title{Forces with KRR}
\author{Malthe Kjær Bisbo}
\date{\today}
\begin{document}
\thispagestyle{empty}
\maketitle

\chapter{Kernel Ridge Regression}
Kernel Ridge Regression (KRR) is basically an upgraded version of regularized linear regression, which is made linear by the kernel trick $x^Tx' \rightarrow k(x,x')$, where $k(.,.)$ is the kernel.

\subsection*{Regularized linear regression}
In regularized linear (least square) regression we seek the coefficients $w$ that minimizes the cost function

\begin{align*}
\min\limits_{w}\left(\sum_{i}^{N}(\vec{w}^Tx_i - y_i)^2 + \lambda\sum_{i}^{N}w_i^2\right)
\end{align*}
or
\begin{align*}
\min\limits_{w}\left((X\vec{w} - \vec{y})^2 + \lambda \norm{\vec{w}}^2\right)
\end{align*}

Where $X$ is the matrix with the $x_i^T$'s as it's rows.

The solution is

\begin{align*}
\vec{w} = X^T(XX^T + \lambda I)^{-1}\vec{y} = X^T\vec{\alpha}
\end{align*}

Where we define $\vec{\alpha} = (XX^T + \lambda I)^{-1}\vec{y}$.

Predictions are then performed using

\begin{align*}
y' = \vec{w}^T x' = (X^T\vec{\alpha})^T x' = \vec{\alpha}^T Xx'
\end{align*}

\bigskip
\textbf{Result:} To train and use the regularized linear regression model as an ML-predictor We thus need two steps.

\bigskip

\textbf{1. Training:}

Calculate $\vec{\alpha}$ from training data.
\begin{align*}
\vec{\alpha} = (XX^T + \lambda I)^{-1}\vec{y}
\end{align*} 

\textbf{2. Prediction:}

Predict new value $y'$ using
\begin{align*}
y' = \vec{\alpha}^T Xx'
\end{align*}

\subsection{Kernel Trick $\rightarrow$ KRR}
Applying the kernel trick using the kernel $k(.,.)$ we get
\begin{align*}
XX^T \rightarrow \mb{K} \quad \text{where} \quad \mb{K}_{ij} = k(\vec{x_i},\vec{x_j})
\end{align*}
and
\begin{align*}
Xx' \rightarrow \vec{\kappa} \quad \text{where} \quad \mb{\kappa}_{i} = k(\vec{x_i},\vec{x'})
\end{align*}

\textbf{Result:} To use \textit{Kernel Ridge Regression} for training and prediction we need the two results.

\bigskip

\textbf{1. Training:}

Calculate $\vec{\alpha}$ from training data.
\begin{align}
\vec{\alpha} = (\mb{K} + \lambda I)^{-1}\vec{y}
\end{align} 

\textbf{2. Prediction:}

Predict new value $y'$ using
\begin{align}
y' = \vec{\alpha}^T \vec{\kappa}
\end{align}

\subsection{Kernel function}

One choice is the \textit{Gaussian kernel}

\begin{align}
k(\vec{x}, \vec{x'}) = \exp\left(\frac{d(\vec{x}, \vec{x'})^2}{2\sigma^2}\right)
\end{align}
where
\begin{align}
d(\vec{x}, \vec{x'}) = \sqrt{\sum_{k}^{M}(x_k - x_k')^2}
\end{align}
With
\begin{align*}
\frac{\partial k}{\partial d} = \frac{d}{\sigma^2}k
\end{align*}
\begin{align*}
\frac{\partial d}{\partial x} = \frac{d}{\sigma^2}k
\end{align*}

\subsection{KRR prediction-error estimate}
The estimate of the prediction error on a new point $x'$ is
\begin{align}
\text{err}(x') = \sqrt{\left|\theta_0[1-\vec{\kappa}(x')\cdot\vec{\alpha_2}(x')]\right|}
\end{align}
where
\begin{align*}
\vec{\alpha_2}(x') = (\mb{K} + \lambda I)^{-1}\vec{\kappa}(x')
\end{align*}
and
\begin{align*}
\theta_0 = \frac{y\cdot \alpha_1}{N_{train}}
\end{align*}


\chapter{Feature}



\end{document}